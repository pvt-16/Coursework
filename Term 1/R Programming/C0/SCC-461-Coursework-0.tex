% Options for packages loaded elsewhere
\PassOptionsToPackage{unicode}{hyperref}
\PassOptionsToPackage{hyphens}{url}
%
\documentclass[
]{article}
\title{SCC 461 Coursework 0}
\author{36071280}
\date{10/10/2021}

\usepackage{amsmath,amssymb}
\usepackage{lmodern}
\usepackage{iftex}
\ifPDFTeX
  \usepackage[T1]{fontenc}
  \usepackage[utf8]{inputenc}
  \usepackage{textcomp} % provide euro and other symbols
\else % if luatex or xetex
  \usepackage{unicode-math}
  \defaultfontfeatures{Scale=MatchLowercase}
  \defaultfontfeatures[\rmfamily]{Ligatures=TeX,Scale=1}
\fi
% Use upquote if available, for straight quotes in verbatim environments
\IfFileExists{upquote.sty}{\usepackage{upquote}}{}
\IfFileExists{microtype.sty}{% use microtype if available
  \usepackage[]{microtype}
  \UseMicrotypeSet[protrusion]{basicmath} % disable protrusion for tt fonts
}{}
\makeatletter
\@ifundefined{KOMAClassName}{% if non-KOMA class
  \IfFileExists{parskip.sty}{%
    \usepackage{parskip}
  }{% else
    \setlength{\parindent}{0pt}
    \setlength{\parskip}{6pt plus 2pt minus 1pt}}
}{% if KOMA class
  \KOMAoptions{parskip=half}}
\makeatother
\usepackage{xcolor}
\IfFileExists{xurl.sty}{\usepackage{xurl}}{} % add URL line breaks if available
\IfFileExists{bookmark.sty}{\usepackage{bookmark}}{\usepackage{hyperref}}
\hypersetup{
  pdftitle={SCC 461 Coursework 0},
  pdfauthor={36071280},
  hidelinks,
  pdfcreator={LaTeX via pandoc}}
\urlstyle{same} % disable monospaced font for URLs
\usepackage[margin=1in]{geometry}
\usepackage{color}
\usepackage{fancyvrb}
\newcommand{\VerbBar}{|}
\newcommand{\VERB}{\Verb[commandchars=\\\{\}]}
\DefineVerbatimEnvironment{Highlighting}{Verbatim}{commandchars=\\\{\}}
% Add ',fontsize=\small' for more characters per line
\usepackage{framed}
\definecolor{shadecolor}{RGB}{248,248,248}
\newenvironment{Shaded}{\begin{snugshade}}{\end{snugshade}}
\newcommand{\AlertTok}[1]{\textcolor[rgb]{0.94,0.16,0.16}{#1}}
\newcommand{\AnnotationTok}[1]{\textcolor[rgb]{0.56,0.35,0.01}{\textbf{\textit{#1}}}}
\newcommand{\AttributeTok}[1]{\textcolor[rgb]{0.77,0.63,0.00}{#1}}
\newcommand{\BaseNTok}[1]{\textcolor[rgb]{0.00,0.00,0.81}{#1}}
\newcommand{\BuiltInTok}[1]{#1}
\newcommand{\CharTok}[1]{\textcolor[rgb]{0.31,0.60,0.02}{#1}}
\newcommand{\CommentTok}[1]{\textcolor[rgb]{0.56,0.35,0.01}{\textit{#1}}}
\newcommand{\CommentVarTok}[1]{\textcolor[rgb]{0.56,0.35,0.01}{\textbf{\textit{#1}}}}
\newcommand{\ConstantTok}[1]{\textcolor[rgb]{0.00,0.00,0.00}{#1}}
\newcommand{\ControlFlowTok}[1]{\textcolor[rgb]{0.13,0.29,0.53}{\textbf{#1}}}
\newcommand{\DataTypeTok}[1]{\textcolor[rgb]{0.13,0.29,0.53}{#1}}
\newcommand{\DecValTok}[1]{\textcolor[rgb]{0.00,0.00,0.81}{#1}}
\newcommand{\DocumentationTok}[1]{\textcolor[rgb]{0.56,0.35,0.01}{\textbf{\textit{#1}}}}
\newcommand{\ErrorTok}[1]{\textcolor[rgb]{0.64,0.00,0.00}{\textbf{#1}}}
\newcommand{\ExtensionTok}[1]{#1}
\newcommand{\FloatTok}[1]{\textcolor[rgb]{0.00,0.00,0.81}{#1}}
\newcommand{\FunctionTok}[1]{\textcolor[rgb]{0.00,0.00,0.00}{#1}}
\newcommand{\ImportTok}[1]{#1}
\newcommand{\InformationTok}[1]{\textcolor[rgb]{0.56,0.35,0.01}{\textbf{\textit{#1}}}}
\newcommand{\KeywordTok}[1]{\textcolor[rgb]{0.13,0.29,0.53}{\textbf{#1}}}
\newcommand{\NormalTok}[1]{#1}
\newcommand{\OperatorTok}[1]{\textcolor[rgb]{0.81,0.36,0.00}{\textbf{#1}}}
\newcommand{\OtherTok}[1]{\textcolor[rgb]{0.56,0.35,0.01}{#1}}
\newcommand{\PreprocessorTok}[1]{\textcolor[rgb]{0.56,0.35,0.01}{\textit{#1}}}
\newcommand{\RegionMarkerTok}[1]{#1}
\newcommand{\SpecialCharTok}[1]{\textcolor[rgb]{0.00,0.00,0.00}{#1}}
\newcommand{\SpecialStringTok}[1]{\textcolor[rgb]{0.31,0.60,0.02}{#1}}
\newcommand{\StringTok}[1]{\textcolor[rgb]{0.31,0.60,0.02}{#1}}
\newcommand{\VariableTok}[1]{\textcolor[rgb]{0.00,0.00,0.00}{#1}}
\newcommand{\VerbatimStringTok}[1]{\textcolor[rgb]{0.31,0.60,0.02}{#1}}
\newcommand{\WarningTok}[1]{\textcolor[rgb]{0.56,0.35,0.01}{\textbf{\textit{#1}}}}
\usepackage{graphicx}
\makeatletter
\def\maxwidth{\ifdim\Gin@nat@width>\linewidth\linewidth\else\Gin@nat@width\fi}
\def\maxheight{\ifdim\Gin@nat@height>\textheight\textheight\else\Gin@nat@height\fi}
\makeatother
% Scale images if necessary, so that they will not overflow the page
% margins by default, and it is still possible to overwrite the defaults
% using explicit options in \includegraphics[width, height, ...]{}
\setkeys{Gin}{width=\maxwidth,height=\maxheight,keepaspectratio}
% Set default figure placement to htbp
\makeatletter
\def\fps@figure{htbp}
\makeatother
\setlength{\emergencystretch}{3em} % prevent overfull lines
\providecommand{\tightlist}{%
  \setlength{\itemsep}{0pt}\setlength{\parskip}{0pt}}
\setcounter{secnumdepth}{-\maxdimen} % remove section numbering
\ifLuaTeX
  \usepackage{selnolig}  % disable illegal ligatures
\fi

\begin{document}
\maketitle

Q1. Write code for log(PCB) against age, reproducing the final plot
containing both the equation line and the data points.

\begin{Shaded}
\begin{Highlighting}[]
\CommentTok{\#Task 1 {-} Reproducing the plot for age vs. log(PCB)}

\NormalTok{ages }\OtherTok{\textless{}{-}} \FunctionTok{seq}\NormalTok{(}\AttributeTok{from=}\DecValTok{0}\NormalTok{, }\AttributeTok{to=}\DecValTok{13}\NormalTok{, }\AttributeTok{by=}\FloatTok{0.1}\NormalTok{)}
\NormalTok{a }\OtherTok{\textless{}{-}} \SpecialCharTok{{-}}\FloatTok{2.3907}
\NormalTok{b }\OtherTok{\textless{}{-}} \FloatTok{2.300}
\NormalTok{l }\OtherTok{\textless{}{-}}\NormalTok{ a }\SpecialCharTok{+}\NormalTok{ b}\SpecialCharTok{*}\NormalTok{ages}\SpecialCharTok{\^{}}\NormalTok{(}\DecValTok{1}\SpecialCharTok{/}\DecValTok{3}\NormalTok{)}
\NormalTok{trout.age }\OtherTok{\textless{}{-}} \FunctionTok{c}\NormalTok{(}\DecValTok{1}\NormalTok{, }\DecValTok{1}\NormalTok{, }\DecValTok{1}\NormalTok{, }\DecValTok{1}\NormalTok{, }\DecValTok{2}\NormalTok{, }\DecValTok{2}\NormalTok{, }\DecValTok{2}\NormalTok{, }\DecValTok{3}\NormalTok{, }\DecValTok{3}\NormalTok{, }\DecValTok{3}\NormalTok{, }\DecValTok{4}\NormalTok{,}\DecValTok{4}\NormalTok{, }\DecValTok{4}\NormalTok{, }\DecValTok{5}\NormalTok{, }\DecValTok{6}\NormalTok{, }\DecValTok{6}\NormalTok{, }\DecValTok{6}\NormalTok{, }\DecValTok{7}\NormalTok{, }\DecValTok{7}\NormalTok{, }\DecValTok{7}\NormalTok{, }\DecValTok{8}\NormalTok{, }\DecValTok{8}\NormalTok{,}
               \DecValTok{8}\NormalTok{, }\DecValTok{9}\NormalTok{, }\DecValTok{11}\NormalTok{, }\DecValTok{12}\NormalTok{, }\DecValTok{12}\NormalTok{, }\DecValTok{12}\NormalTok{)}
\NormalTok{trout.pcb }\OtherTok{\textless{}{-}} \FunctionTok{c}\NormalTok{(}\FloatTok{0.6}\NormalTok{ , }\FloatTok{1.6}\NormalTok{, }\FloatTok{0.5}\NormalTok{, }\FloatTok{1.2}\NormalTok{, }\FloatTok{2.0}\NormalTok{, }\FloatTok{1.3}\NormalTok{, }\FloatTok{2.5}\NormalTok{,}\FloatTok{2.2}\NormalTok{, }\FloatTok{2.4}\NormalTok{, }\FloatTok{1.2}\NormalTok{, }\FloatTok{3.5}\NormalTok{, }\FloatTok{4.1}\NormalTok{, }\FloatTok{5.1}\NormalTok{, }\FloatTok{5.7}\NormalTok{,}
               \FloatTok{3.4}\NormalTok{, }\FloatTok{9.7}\NormalTok{, }\FloatTok{8.6}\NormalTok{, }\FloatTok{4.0}\NormalTok{, }\FloatTok{5.5}\NormalTok{, }\FloatTok{10.5}\NormalTok{, }\FloatTok{17.5}\NormalTok{,}\FloatTok{13.4}\NormalTok{, }\FloatTok{4.5}\NormalTok{, }\FloatTok{30.4}\NormalTok{, }\FloatTok{12.4}\NormalTok{, }\FloatTok{13.4}\NormalTok{, }\FloatTok{26.2}\NormalTok{, }\FloatTok{7.4}\NormalTok{)}
\FunctionTok{plot}\NormalTok{(}\AttributeTok{x=}\NormalTok{trout.age, }\AttributeTok{y=}\FunctionTok{log}\NormalTok{(trout.pcb), }\AttributeTok{main=}\StringTok{"Log(PCB) vs Age"}\NormalTok{, }\AttributeTok{xlab =} \StringTok{"Age {-}\textgreater{}"}\NormalTok{, }\AttributeTok{ylab =} \StringTok{"Log(PCB) {-}\textgreater{}"}\NormalTok{)}
\FunctionTok{lines}\NormalTok{(}\AttributeTok{x=}\NormalTok{ages, }\AttributeTok{y=}\NormalTok{l, }\AttributeTok{type=}\StringTok{"l"}\NormalTok{, }\AttributeTok{col=}\StringTok{"blue"}\NormalTok{)}
\end{Highlighting}
\end{Shaded}

\includegraphics{SCC-461-Coursework-0_files/figure-latex/unnamed-chunk-1-1.pdf}

Q2. Rewrite the log(PCB) equation as a function which has arguments; a,
b, and age, and returns the predicted log(PCB).

\begin{Shaded}
\begin{Highlighting}[]
\CommentTok{\#Task 2 {-} Rewrite log(PCB) as a function}

\NormalTok{calc\_PCB }\OtherTok{=} \ControlFlowTok{function}\NormalTok{(a, b, age) \{}
\NormalTok{  predicted\_l }\OtherTok{\textless{}{-}}\NormalTok{ a }\SpecialCharTok{+}\NormalTok{ b}\SpecialCharTok{*}\NormalTok{age}\SpecialCharTok{\^{}}\NormalTok{(}\DecValTok{1}\SpecialCharTok{/}\DecValTok{3}\NormalTok{)}
  \FunctionTok{return}\NormalTok{ (predicted\_l)}
\NormalTok{\}}
\end{Highlighting}
\end{Shaded}

Q3. By extending the range of age considered, produce a plot which shows
the curve for the expected log(PCB) concentration for lake trout up to
20 years old.

\begin{Shaded}
\begin{Highlighting}[]
\CommentTok{\#Task 3 {-} Extend ages to 20 years}

\NormalTok{ages\_20 }\OtherTok{\textless{}{-}} \FunctionTok{seq}\NormalTok{(}\AttributeTok{from=}\DecValTok{0}\NormalTok{, }\AttributeTok{to=}\DecValTok{20}\NormalTok{, }\AttributeTok{by=}\FloatTok{0.1}\NormalTok{)}
\FunctionTok{plot}\NormalTok{(}\AttributeTok{x=}\NormalTok{ages\_20, }\AttributeTok{y=}\FunctionTok{calc\_PCB}\NormalTok{(a,b,ages\_20), }\StringTok{"l"}\NormalTok{, }\AttributeTok{col=}\StringTok{"green"}\NormalTok{, }\AttributeTok{main=}\StringTok{"Log(PCB) vs Age for 20 years"}\NormalTok{, }
     \AttributeTok{xlab=}\StringTok{"Age {-}\textgreater{}"}\NormalTok{, }\AttributeTok{ylab=}\StringTok{"Log(PCB) {-}\textgreater{}"}\NormalTok{)}
\end{Highlighting}
\end{Shaded}

\includegraphics{SCC-461-Coursework-0_files/figure-latex/unnamed-chunk-3-1.pdf}

Q4. Now extract the maximum expected/predicted log(PCB) from the values
used to draw the equation line

\begin{Shaded}
\begin{Highlighting}[]
\CommentTok{\#Task 4 {-} Get Maximum}
\FunctionTok{print}\NormalTok{(}\FunctionTok{paste0}\NormalTok{(}\StringTok{"Maximum value is :"}\NormalTok{, }\FunctionTok{max}\NormalTok{(}\FunctionTok{calc\_PCB}\NormalTok{(a,b,ages\_20))))}
\end{Highlighting}
\end{Shaded}

\begin{verbatim}
## [1] "Maximum value is :3.85246051816828"
\end{verbatim}

Q5. It can be shown that a non-linear model of the form l = a + b ×
age\^{}c where a, b, and c are constants provides a slightly better fit
to the data. The optimal choices are a = -4.865, b = 4.7016, and c =
0.1969.

\begin{Shaded}
\begin{Highlighting}[]
\CommentTok{\#Task 5 {-} Non{-}linear model}

\NormalTok{a2 }\OtherTok{=} \SpecialCharTok{{-}}\FloatTok{4.865}
\NormalTok{b2 }\OtherTok{=} \FloatTok{4.7016}
\NormalTok{c2 }\OtherTok{=} \FloatTok{0.1969}
\end{Highlighting}
\end{Shaded}

\begin{enumerate}
\def\labelenumi{(\alph{enumi})}
\tightlist
\item
  Rewrite the log(PCB) equation as a function which has arguments; a, b,
  c and age, and returns the predicted log(PCB).
\end{enumerate}

\begin{Shaded}
\begin{Highlighting}[]
\CommentTok{\# Task 5.1 {-} Rewrite log equation function}

\NormalTok{calc\_PCB2 }\OtherTok{=} \ControlFlowTok{function}\NormalTok{(a,b,c,age) \{}
\NormalTok{  predicted\_l }\OtherTok{=}\NormalTok{ a }\SpecialCharTok{+}\NormalTok{ b }\SpecialCharTok{*}\NormalTok{ (age}\SpecialCharTok{\^{}}\NormalTok{c) }
  \FunctionTok{return}\NormalTok{ (predicted\_l)}
\NormalTok{\}}
\end{Highlighting}
\end{Shaded}

\begin{enumerate}
\def\labelenumi{(\alph{enumi})}
\setcounter{enumi}{1}
\tightlist
\item
  Compare the Bates-Watts estimator and the new estimator for the
  expected log(PCB) concentration of a 10 year old lake trout.
\end{enumerate}

\begin{Shaded}
\begin{Highlighting}[]
\CommentTok{\# Task 5.2 {-} Compare}

\NormalTok{fixed\_age }\OtherTok{=} \DecValTok{10}
\NormalTok{value\_BWEstimator }\OtherTok{=} \FunctionTok{calc\_PCB}\NormalTok{(a,b,fixed\_age)}
\NormalTok{value\_newEstimator }\OtherTok{=} \FunctionTok{calc\_PCB2}\NormalTok{(a2,b2,c2,fixed\_age)}

\NormalTok{value\_newEstimator }\SpecialCharTok{==}\NormalTok{ value\_BWEstimator}
\end{Highlighting}
\end{Shaded}

\begin{verbatim}
## [1] FALSE
\end{verbatim}

\begin{enumerate}
\def\labelenumi{(\alph{enumi})}
\setcounter{enumi}{2}
\tightlist
\item
  Create a new plot which has both the old line and new line, allowing a
  comparison of the differences.
\end{enumerate}

\begin{Shaded}
\begin{Highlighting}[]
\CommentTok{\# Task 5.3 {-} Plot both equations}

\FunctionTok{plot}\NormalTok{(}\AttributeTok{x=}\NormalTok{ages, }\AttributeTok{y=}\NormalTok{l, }\AttributeTok{type=}\StringTok{"l"}\NormalTok{, }\AttributeTok{col=}\StringTok{"red"}\NormalTok{, }\AttributeTok{ylab =} \StringTok{"Log(PCB) {-}\textgreater{}"}\NormalTok{, }\AttributeTok{xlab =} \StringTok{"Ages {-}\textgreater{}"}\NormalTok{, }
     \AttributeTok{main=}\StringTok{"Comparing Bates{-}Watts estimator and the new estimator"}\NormalTok{)}
\FunctionTok{lines}\NormalTok{(ages, }\FunctionTok{calc\_PCB2}\NormalTok{(a2,b2,c2,ages), }\AttributeTok{type=}\StringTok{"l"}\NormalTok{, }\AttributeTok{col=}\StringTok{"orange"}\NormalTok{)}
\end{Highlighting}
\end{Shaded}

\includegraphics{SCC-461-Coursework-0_files/figure-latex/unnamed-chunk-8-1.pdf}

\end{document}
